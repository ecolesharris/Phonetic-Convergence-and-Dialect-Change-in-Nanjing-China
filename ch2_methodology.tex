\chapter{Methodology}
% * <ecolesharris@gmail.com> 2018-06-19T19:42:08.609Z:
% 
% > Methodology
% Kayla said: One critique I would make is on your organization- I think it would be easier for the reader if you explain things clearly earlier (as you first introduce things) vs later (when they are necessairy to explain because of data interp )- in this way you are setting your reader up to understand better what you are about to lay down for them
% 
% ^.
\label{methodchapter}
In the following chapter, I will lay out the procedures I followed for subject selection, the experiment, and data annotation.

\section{Participant Selection}
\label{sec:subjsel}
Fifty participants were enrolled in and completed the study. Eligible participants were between the ages of 18 and 35, and self-reported as being fluent in one of the dialects of interest.
% * <ecolesharris@gmail.com> 2018-06-19T19:42:52.297Z:
% 
% > dialects of interest
% Kayla said: does it matter which one?
% 
% ^ <ecolesharris@gmail.com> 2018-06-19T19:46:56.752Z:
% 
% No, don't believe so. I'm going to be looking at their relative dominance in each dialect (as measured by the BLP) later, but that's only after they've met the inclusion criteria.
%
% ^.
% * <ecolesharris@gmail.com> 2018-06-19T19:29:28.304Z:
% 
% > self-reported
% Kayla said: is self-reported really a good measurement of being fluent? This might be a hole in accuracy - were they subjected to a test or did someone confirm their fluency?- there could be a claim here that this would infleunce your data results- how will you deal with that?
% 
% ^ <ecolesharris@gmail.com> 2018-06-19T19:45:56.902Z:
% 
% Self-reporting should be okay here, cuz I think there's no reason to disbelieve a participant that says either "yes, I'm fluent" or "no, I'm not." 
%  Since it's just an inclusion criteria and I'm not using like a measure of degree of fluency to try to predict anything, it shouldn't really be a big deal for the results.
%
% ^.
They also had at least enough fluency in the other dialect to read a wordlist with native-like pronunciation. No participants reported having  hearing loss or speech abnormalities at the time of selection. Age and self-reported fluency were the only inclusion criteria; abnormal speech or hearing were the only exclusion criteria.% In addition to the selection criteria described above, demographic information including gender, education level, district of residence within Nanjing, and district of which the subject’s speech is most representative were collected for each subject.
% * <ecolesharris@gmail.com> 2018-06-19T19:47:36.061Z:
% 
% > abnormal hearing or speech
% Kayla said: As in a disability? or something wrong with the recording… please specify.
% 
% ^.

Participants were recruited in one of three ways: 1) I approached them directly and asked about their ability to speak ND and PTH, and whether they would be interested in participating in an experiment; 2) a participant or potential participant contacted them and referred them to me; or 3) they responded to an online advertisement posted on the popular Chinese social media platform, WeChat. The largest portion of participants were recruited via this last means, just under two thirds of them. Referrals constituted the next largest source, yielding roughly one third of all participants. My approaching potential participants in person only resulted in a few people participating in the experiment---three, to be exact. 
% * <ecolesharris@gmail.com> 2018-06-19T19:52:34.047Z:
% 
% > exact
% Kayla said: This paragraph could be a lot more efficiently worded
% 
% ^.
% * <ecolesharris@gmail.com> 2018-06-19T19:51:14.763Z:
% 
% > I
% Kayla said: How 1st person should this be?
% 
% ^.

\section{Experimental Procedure}
\label{sec:procedure}
The experimental procedure consisted of four separate sub-procedures: a dialect dominance survey called the Bilingual Language Profile (BLP; \S\ref{ssec:procedureBLP}; \citealt{birdsong2012bilingual}), a shadowing experiment (\S\ref{ssec:procedureShadowing}), a test of implicit social biases called an Implicit Associations Task (IAT;\S\ref{ssec:procedureIAT}; \citealt{greenwald1998measuring,greenwald2003understanding}), and an interview (\S\ref{ssec:procedureInterview}). The entire procedure tended to take roughly one hour and thirty minutes, and participants were compensated with RMB \textyen150 for their time.  The BLP, shadowing experiment, and IAT were all conducted with the subject seated at a laptop computer. For the BLP, questions were presented and responses were collected using Google Forms. For the shadowing experiment and IAT, stimuli were presented by the psychological experimentation software PsychoPy (\cite{peirce2007psychopy}, \cite{peirce2009generating}), and for the IAT, participant response reaction times were recorded by PsychoPy using a specially modified, millisecond-accurate keyboard from Empirisoft. The shadowing and interview portions of the experiment were recorded using an Audio-Technica shotgun microphone recording 16-bit .WAV files monaurally into a Tascam DR-40 Linear PCM Recorder at 44.1kHz. All but one recording session took place at the Nanjing University Language Education Comprehensive Laboratory. The recording session that did not take place there was instead conducted in a student lounge in an academic building on the Nanjing University Gulou campus.

\subsection{Bilingual Language Profile}
\label{ssec:procedureBLP}
Once participants were identified and provided consent to participate in the experiment, I had them fill out a translated and modified version of the BLP online using Google Forms. This survey first collects basic demographic information including age, gender identity, current district of residence within the city, and district of which the subject’s speech is most representative, and then asks 19 questions designed to assess the relative dominance of PTH and ND within a single speaker. See Appendix A (pg. \pageref{appendix:BLP}) for the English text of this survey.

The BLP is divided into four sections that compare PTH and ND: participant history, use, proficiency, and attitudes. For each category, the BLP calculates a numeric score for each of the dialects of interest. The sum of the PTH scores for all four categories is then subtracted from the sum of the ND scores, yielding a “dominance score” consisting of a number between -217.94 and 217.94. The negative pole in this case indicates complete PTH dominance, and the positive pole indicates complete ND dominance.

All 50 participants completed every question in the survey. Subjects ranged in age from 18 years old to 35 years old, with an average age of 24.4. Twenty-eight subjects identified as women, and twenty-two as men. All subjects had at least attended some college, and several had completed advanced degrees. Six districts around the city were represented in the district of residence data; five of those overlapped with the representative dialect districts, of which there were seven.

The overall trend in the BLP data was towards PTH dominance. The mean score across all participants was -25.68, indicating slight PTH dominance in the sample as a whole. Forty subjects were on the PTH-dominant side of the scale, and 10 were on the ND-dominant side. The strongest individual PTH-dominant score was -105.44, while the strongest ND-dominant score was only 36.78. These results are summarized in Table \ref{table:BLPResults}.

\begin{table}
\centering
\begin{tabular}{|c||c|c|c|}
 \hline
  & PTH Dominant Subjects & Overall & ND Dominant Subjects \\ [0.5ex] 
 \hline\hline
 Number & 40 & 50 & 10 \\ 
 \hline
 Mean & -36.09 & -25.68 & 15.96 \\
 \hline
 Extreme & -105.44 & N/A & 36.78 \\
 \hline
\end{tabular}
\caption{Some BLP Results}
\label{table:BLPResults}
\end{table}

\subsection{Shadowing}
\label{ssec:procedureShadowing}
After completing the BLP, I used a shadowing paradigm to attempt to induce convergence with two dialectal features---one the product of acoustic manipulation and the other a naturally occurring variant---into participants’ PTH. In addition to shadowing these two features, baseline and posttest recordings were made of words displaying three other dialectal features to determine whether exposure to PTH encouraged change even in features that the participants were not shadowing.

\paragraph{Production List.}
\label{para:prodList}
The production list was designed to elicit 30 tokens each of five speech features. All five features, their statuses as Labovian markers or stereotypes (\hl{CITE LABOV 1971 The study of language in its social context}), and whether or not they are shadowed in the experiment are listed in Table \ref{table:shadowingFeatures}. Each of these five features is gradiently measurable via acoustic means, giving me the ability to correlate measured values with things like age, language dominance, and degree of convergence using linear mixed-effects modeling. The production list can be found in its entirety in Appendix B, starting on pg. \pageref{appendix:ProductionList}.

\begin{table}
\centering
 \begin{tabular}{|c|c|c|} 
 \hline
 Feature & Status & Shadowed?  \\ [0.5ex] 
 \hline\hline
 Extended [p\super{h}] & Novel & Yes \\ 
 \hline
 ND [\~{\ae}] vs. PTH [\ae n] & Marker & Yes \\
 \hline
 ND [e] vs. PTH \textipa{[iE]} & Marker & No \\
 \hline
 ND [i] vs. PTH [y] & Stereotype & No \\
 \hline
 ND [l] vs. PTH [n] & Stereotype & No \\
 \hline
\end{tabular}
\caption{Shadowing \& Mandarinization Features}
\label{table:shadowingFeatures}
\end{table}

Of the five variables, one---extended [p\super{h}]---was meant to be an entirely novel dialectal feature. That is to say, at the time the experiment was designed length of VOT was not known to be distinct between the two dialects, and so any association between PTH and long VOT on [p\super{h}] would necessarily be the product of exposure to that variable as part of the shadowing experiment. In addition to being novel, VOT is easily measurable and manipulable using sound editing software, and extended VOT has been shown to be both imitable (\cite{shockley2004imitation}) and persistent within a dialect (\cite{nielsen2008word}), and so it was an ideal feature for experimentation.

Each of the other four features is a preexisting dialectal variable that differs between ND and PTH, but the PTH forms of which are gradually being adopted into ND (\cite{bao1980sixty}---a process that I call ``Mandarinization''. These four Mandarinization features also provide a way to assess the degree of advancement of dialect shift for each speaker, which has been shown to correlate negatively with willingness to imitate (\hl{cite Pinget 2015}). The four features are ND [\~{\ae}] vs. PTH [\ae n], ND [e] vs. PTH \textipa{[iE]}, ND [i] vs. PTH [y], and ND [l] vs. PTH [n].  %To clarify, when native speakers of ND use either ND or PTH to produce syllables that in perfectly standard PTH would be produced with the rhyme [\ae n], they tend to actually produce [\~{\ae}].

Extended [p\super{h}] and ND [\~{\ae}] vs. PTH [\ae n] were chosen as exposure variables, meaning that participants would be exposed to and asked to shadow auditory stimuli that displayed these features during the shadowing experiment. Extended [p\super{h}] was chosen, as mentioned above, for its novelty and ease of measurement. ND [\~{\ae}] vs. PTH [\ae n] was chosen as the other exposure variable in large part because this difference between ND and PTH is not, as far as I know, socially salient to native speakers of ND, and so should be comparable to the extended [p\super{h}] stimuli in terms avoiding variation due to variable social judgments. Anecdotally, I have had a small number of conversations with ND speakers that suggest they are entirely unaware of the existence of this variable.

%The variables extended [p\super{h}] and ND [\~{\ae}] vs. PTH [\ae n], were chosen as exposure variables for a number of reasons. First, extended VOT on the phoneme /p\super{h}/ was meant to be an entirely novel dialectal feature, in that

In order to create the extended [p\super{h}] stimuli that participants would be exposed to, I hired a consultant who is fluent in both ND and PTH and asked her to produce all of the words for that feature with normal levels of aspiration. I then edited the recordings by duplicating and reinserting the central 50\% of the aspiration into the original recording twice, effectively doubling each token’s VOT. This editing of stimuli was done in Praat (\cite{praat}). To create the ND [\~{\ae}] vs. PTH [\ae n] stimuli, I asked my consultant to produce the words for this feature from the production list naturally, but to be careful to include a nasal coda; I then double-checked that each word did in fact include a nasal coda by examining a spectrogram of the recording in Praat.

\paragraph{Experimental Procedure.}
\label{para:shadowingProcedure}
The shadowing experiment consisted of four blocks: an ND pretest block, a PTH pretest block, a PTH shadowing test block, and an ND posttest block. All participants completed the four blocks in this order, with the ND blocks on either end of the experiment to minimize the priming effects of PTH before the test block and maximize it afterwards.

Blocks 1 and 2 of the shadowing portion of the procedure were designed to elicit baseline pronunciations of all of the words in the production list in ND and PTH, respectively. At the beginning of the ND and PTH pretest blocks, the following instructions appeared on the laptop screen: ``During this part of the experiment, the computer will randomly display Chinese characters. Please use [dialect] to read these characters out loud. When you're ready, please press the space bar to continue.'' Once the participant dismissed the instructions, words began to appear in the center of an otherwise blank screen, one word every two seconds. During these blocks, subjects read all of the words from the production list in randomized order, yielding 150 tokens for each of the first two blocks.  

At the beginning of the PTH exposure block, participants donned a pair of Beats Solo 3 headphones and the following instructions appeared: ``During this part of the experiment, the computer will randomly play recordings of PTH words. Some recordings will repeat, this is normal. Please quickly use PTH to clearly repeat the words you hear. When you're ready, please press the space bar to continue.'' Once these instructions were dismissed, the extended [p\super{h}] and [\~{\ae}] vs. [\ae n] stimuli began to play over the headphones in random order, one word every two seconds. The auditory stimuli were not accompanied by any text on the laptop screen. There were 60 stimuli presented in this block, 30 for each feature. Following Zellou, Scarborough and Nielsen (\cite{zellou2016phonetic}), each of the 60 stimuli was played twice, for a total of 120 stimulus presentations. 

%It is likely that some of these features will be accommodated to more than others. Trudgill (\cite{trudgill1986dialects}) claims that it is salient features that are accommodated to, as long as they’re not too salient, while Babel (\cite{babel2012evidence}) suggests that phonetic space is a factor in convergence. With my interview data, I will be able to assess which of these variables are salient to my participants, and because each is gradiently measurable, I will be able to assess whether phonetic space plays any role.

The final block of the shadowing experiment was identical to the ND pretest block. The entire production list was presented in random order as text stimuli on an otherwise blank screen.

\paragraph{Technical difficulties.}
\label{para:shadowingTechDiffs}
Due to some technical hiccups, not all recording sessions went exactly according to plan. Specifically,
%Subject 17
during one recording session, Psychopy crashed after presenting only 84 stimuli in the PTH shadowing block, so I had the participant start that block over and continue with the experiment. %did I end up using any of the tokens from before the crash, or only after?
%Subject 8
During another recording session PsychoPy crashed 19 stimuli into the ND pretest block, and so I had the participant start the sub-procedure over. Later during that same session, the recorder crashed briefly in the middle of the ND posttest, so 31 tokens are missing from that participant's recording.

\paragraph{Annotation}
\label{para:shadowingAnnotation}
The data collected during the shadowing portion of the experiment were annotated in Praat (\citep{praat}). Specifically, I measured durations of VOTs for the extended [p\super{h}] words, duration of nasal coda %and degree of vowel nasality
for the [\~{\ae}] vs. [\ae n] words, degree of diphthongization for the [e] vs. \textipa{[iE]} words, and formant structure (F1, F2, F3) at the midpoint of the vowels in the [i] vs. [y] words and the segments of interest in the [l] vs. [n] words.

Initially, recordings were annotated to Praat TextGrids and intervalized by hand. They were divided into intervals of roughly 30 seconds each, and all of the elicited words in within those 30 second intervals were either hand-typed in or copied and pasted from a CSV created by PsychoPy. The intervalized TextGrids were then run through the Montreal Forced Aligner (MFA, \cite{mcauliffe2017montreal}), which aligned separate intervals for each word and phone in the recording. The newly-aligned TextGrids were then run through an interval selector that copied the phone interval For each segment of interest to a target tier, creating new, separate TextGrid for each subject for each feature being examined.

VOT duration measurements for the extended [p\super{h}] phones and nasal coda durations for the ND [\~{\ae}] vs. PTH [\ae n] phones were hand-corrected based on the forced aligner's output. %I used 4 clues to decide where to align formant boundaries for the beginning of codas; 1) loss of clear formant structure at F2 or above, 2) drop (usually) in amplitude from preceding vowel (or rise in amplitude if the vowel was so creaky it devoiced, as sometimes happened), 3) simplification of the waveform, and 4) auditory cues (i.e., "did it sound like there was a nasal starting right there to me?"). I just lined up the coda ending boundaries with the end of the visible pitch track or sometime formants (I know that's not great, but I felt like I needed some kind of rule to follow, and that's what I came up with. ALSO! Should I throw out all shizhan? I feel like I should, since a lot of people seem to have thought that it was "shizheng." ALSO! I seem to have been inconsistently weighting the change in formant structure vs. the way that the segment sounds. I think *mostly* I'd been following the formants, but recently (like maybe in recordings 22 & 23) I'd been thinking more about when it *sounds* like the nasal coda starts. I'm gonna try to go back to formants now. Rebecca pointed out on 2018/07/17 that the duration measure might not be super reliable anyway. ALSO, I need to remember whether I was including [m] codas for tokens before subj 25 interval 182. I will from then on, but I don't know if I did before. Also, in subject27, I decided that things that sounded like approximants rather than full nasal stops didn't count as codas. I don't know if I did that with other people.
F1, F2, and F3 formant values were automatically collected at the midpoints of the ND [i] vs. PTH [y] phones and the ND [l] vs. PTH [n] phones by means of a Praat script written by Will Styler and tweaked by Rebecca Scarborough and myself.

There were a butt-load of spurious F3 values in the [i]/[y] data (some well over 4000 Hz), which I hand-corrected by finding all instances of F3 that were more than 115\% of the mean value of F3 for the entire dataset (mean: 2658 Hz, 115\% of mean: 3057 Hz). This approach yielded 292 vowels to be corrected. To get more reliable measurements, I used Praat's "Formant Listing" function at a point in those vowels that closely approximated the formant values at their midpoints, and updated F1, F2, and F3 according to the new reading. During this process, I found \hl{XXXXX} vowel tokens that I determined to be unmeasurable, even by hand, and so these tokens were deleted from the dataset. 

\subsection{Implicit Associations Task}
\label{ssec:procedureIAT}
After the shadowing task, subjects were asked to complete an Implicit Associations Task (IAT), wherein they sorted stimuli into one of four categories---good, bad, PTH, or ND. The IAT used for this study was based on descriptions of other IATs found in \cite{greenwald1998measuring} and \cite{greenwald2003understanding}.

\paragraph{IAT Stimuli.}
\label{para:IATstims}
The good and bad stimuli used in this IAT were my translations of the "positive words" and "negative words" used in experiments 1 and 2 of \cite{greenwald1998measuring}. These stimuli were presented solely visually as text on a computer screen.

The PTH and ND stimuli were selected in consultation with a native speaker of both PTH and ND according to the following selection criteria:
\begin{itemize}
\item The word must be relatively free of evaluative judgments, so as not to be confused with the good/bad stimuli.
\item It must be possible to represent the word in Chinese characters, so as to make the semantics of the character clear to participants to avoid confusion with homophones.
\item If the word exists in both dialects, the pronunciation of the word must be distinct between the dialects, so that the dialect can be determined from the combination of the phonetics and the semantics.
\end{itemize}
The PTH and ND stimuli were presented simultaneously as audio and text. It was necessary to present these stimuli as audio because, in the absence of greater linguistic context, discriminating between the two dialects is primarily accomplished by phonetic means. It was necessary to accompany the audio with a written representation of the word to aid in dialect recognition by avoiding potential semantic ambiguities that might make it impossible to distinguish between the dialects. 

\paragraph{IAT Procedure}
\label{para:IATprocedure}
The task was divided into seven blocks with differing subsets of stimuli being presented during each block, as listed in Table \ref{table:IATprocedure}. For half the subjects, the blocks were completed in the order presented in Table \ref{table:IATprocedure}. For the other half, the positions of Blocks 1, 3, and 4 were switched with those of Blocks 5, 6, and 7, respectively. Also, the procedure in Blocks 3, 4, 6, and 7 was to alternate trials that present either a good or a bad word with trials that present either a PTH or ND word. This was meant to overcome order effects that are known to be an issue with IATs \citep{greenwald2003understanding}. Button press reaction times were collected during this task to assess subjects’ relative implicit preference for either PTH or ND.

\begin{table}
\centering
 \begin{tabular}{|c|c|c|c|c|} 
 \hline
 Block & No. of Trials & Function & Left Key Stimuli & Right Key Stimuli \\ [0.5ex] 
 \hline\hline
 1 & 20 & Practice & PTH Words & ND Words\\ 
 \hline
 2 & 20 & Practice & Good Words & Bad Words\\
 \hline
 3 & 20 & Practice & Good + PTH Words & Bad + ND Words\\
 \hline
 4 & 40 & Test & Good + PTH Words & Bad + ND Words\\
 \hline
 5 & 20 & Practice & ND Words & PTH Words\\
 \hline
 6 & 20 & Practice & Good + ND Words & Bad + PTH Words\\
 \hline
 7 & 40 & Test & Good + ND Words & Bad + PTH Words\\
 \hline
\end{tabular}
\caption{IAT procedure}
\label{table:IATprocedure}
\end{table}

In blocks 1, 2, and 5, participants were presented with stimuli from two of the four categories, requiring them to discriminate between the two target-concepts, PTH and ND, or between the two poles of the associated attribute, good and bad. In blocks 3, 4, 6, and 7, participants were presented with stimuli from all four categories, requiring them to discriminate between good/bad and PTH/ND at the same time. Two of these categories were associated with each response key---for example good and PTH with the left key, bad and ND with the right. See the table above for a list of which categories were active in which blocks, and which response key they were assigned to. By requiring participants to discriminate between good/bad and PTH/ND simultaneously and measuring their reaction times, we get a measure of how strongly the categories that are associated with a single key are associated in the participants’ minds. That is to say, if participants are quicker to respond when good and PTH are associated with a single key than when good and ND are associated with a single key, the participant’s mental association of goodness with PTH is stronger than their association of goodness with ND. In this way we can assess implicit goodness judgments for the two dialects of interest, predicting convergence and potentially generalization of dialectal features. See Appendix C for a complete list of IAT stimuli.

No annotation of the raw IAT reaction time data was needed. That data was used to calculate an “IAT Score” for each subject using the "improved algorithm" for D-score calculation described by \cite{greenwald2003understanding}. Positive scores indicate preference for Mandarin over Nanjing Dialect, and negative scores the reverse.%based on a note under Table 1 in Greenwald and whoever 1998 (pp 1469)

\subsection{Interview}
\label{ssec:procedureInterview}
Finally, after the IAT, Subjects were asked to complete a short semi-structured interview. The idea behind this section of the study was to assess explicit attitudes towards the two dialects. See Appendix D (XXXXX link) for a list of pre-planned interview questions.

While I am not certain at this point that I will be using the interview data for this dissertation, if I do, I will annotate the data using roughly the following process. First, it will be necessary to transcribe the interviews in the language they were conducted in. For the most part, that was PTH, though several participants decided to respond in ND, and there was occasional code switching into English in a number of the interviews. Once transcribed, I will distill from the data a number of recurring themes (e.g., ``ND is for intimate relationships,'' ``PTH is the default language for public interaction,'' ``allow your interlocutor to determine the language of interaction'') and create a spreadsheet that codes for each of these themes in every interview. Once I have this representation of the basic ideas present in the interviews, I can use it to describe general quantitative patterns in how Nanjingers feel about both ND and PTH---ideally being able to assign either a “positive” or “negative” value to their attitudes regarding PTH and ND---as well as to locate what may be useful quotes for conducting more qualitative analyses.

%I double-checked and hand-corrected outliers (upper quartile of block + 1.5 x SD of block) for LN. There were 81 of these, plus two that I threw out cuz I didn't think they counted.

%\section{Annotation and Measurement}
%\label{sec:annotation}
%The annotation and measurement of the BLP (\S\ref{ssec:annotationBLP}), shadowing experiment data (\S\ref{ssec:annotationShadowing}), IAT data (\S\ref{ssec:annotationIAT}), and interview data (\S\ref{ssec:annotationInterview}) are described below.

%Due to technical hiccup (i.e. I neglected to change something before starting to run subjects) at least the first five subjects say pindi instead of pinde in the baseline and posttest conditions.
%
%I need to take bieguan and bieku out of the [e] \textipa{[iE]} analysis, cuz 1)sometimes people switch bie for biao, 2) it's too hard to decide systematically which ones should be in or out, and 3) taking them all out is not that huge a loss to the data.
%Forgot to include nieshouniejiao in the LN words, only used it in the ND [e] vs. PTH \textipa{[iE]} words.
%
%\subsection{Bilingual Language Profile}
%\label{ssec:annotationBLP}
%Blah Blah Blah
%
%\subsection{Shadowing}
%\label{ssec:annotationShadowing}
% This set of five variables will be referred to as the “Mandarinization variables.”
%
%
%\subsection{Implicit Associations Task}
%\label{ssec:annotationIAT}
%Blah blah blah.
%
%\subsection{Interview}
%\label{ssec:annotationInterview}

%BUT WAIT!!! There were a few very long rt latencies (the longest being 24,699 ms), and that paper says they "somewhat arbitrarily" removed latencies longer than 10,000 ms. I haven't done that. I should talk to a psych statistician maybe and be like "how can I know which of these, if any, should be excluded?" I was thinking I could be like "I'll delete 'outliers' from the data, but then there's the problem of how you define an outlier. I could delete any data point more than 1.5 IQRs below the first quartile or above the third quartile, which would mean I'd delete any value greater than 2,227 ms or less than -106 ms (which is obviously impossible). That would mean deleting the 343 longest reaction times. At one point in the article they also make mention of excluding fast (<300ms) and slow (>3,000ms) responses. I could do that, or even recode the >3,000ms responses to be exactly 3,000ms like they did in Greenwald et al. 1998...There's also the question of whether to exclude any participants from the analysis. In Greenwald et al. 2003, they try excluding people with more than 10% fast responses (<300ms) but I don't have even one single response that is fast by those standards. They also talk about excluding people with unusually high error rates in block 2,3,4,6, and 7; I have one person with an error rate of 6.7% (subject8) and one with an error rate of 5.8% (subject26) and 4 with an error rate of 4.2%... I was thinking *maybe* exclude people with error rates higher than 5%, but that's kind of arbitrary. Rebecca suggests thinking about stdev because they're easy and psych lit sometimes does throw out 2-3 sd's above/below. Rebecca thinks spending a long time on figuring out influential point analysis is a waste of time.





\pagebreak
this page intentionally left blank
\pagebreak









