\chapter{Discussion}
\label{discussionchapter}

It seems to me that I don't have good evidence for a direct, mechanistic link between interpersonal accommodation and real-world sound change, at least not one like Trudgill suggests. In my opinion, the fact that different variables behave differently in the experiment argues against that model, at least the model that says ``the more numerous speakers will just overwhelm the less numerous ones.'' I don't have solid numbers about how numerous the speakers of the dialects are across the city, but consider that:

\begin{enumerate}
    \item there's a Mandarin dominance bias in my subject pool, I'm pretty sure both in terms of the BLP results \textit{and} the IAT results;
    \item anecdotally, people who don't speak any Nanjing Dialect in the city WAAAAY out number people who don't speak any Mandarin.
\end{enumerate}

I think ``mechanistic'' is way too strong a word. ``Probabilistic'' is almost certainly better, but i don't really have a strong reason to argue for that particular term.

It's true that the way my experiment works is not exactly the same way that accommodation works in the real world, but it seems unlikely to me that the relationship between lab accommodation and real-world sound change should be the exact opposite of real-world accommodation and real-world sound change.



\pagebreak

This page intentionally left blank.