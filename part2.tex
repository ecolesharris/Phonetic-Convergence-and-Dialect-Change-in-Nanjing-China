\part{Part 2}
\label{part2}
%\hl{I used to frame this as a }``\cbat{} \hl{vs. identity projection theory'' thing. I seem to have forgotten all about that. Does that framework work here? Is it helpful? [R: Prolly more helpful to be like ``this is a test of one implementation of }\cbat{}\hl{. You don't have good evidence to prove it's Ident Proj in any case.]}

    In the next two chapters, I'm going to be addressing the \textipa{[g\textturnr E::t\textcorner]} big question of whether it is interpersonal accommodation that drives contact-induced language change (it's not) and if not (it's not) what is? The two nitty-grittier questions I'm going to be answering over the next two chapters are:
    \begin{enumerate}
        \item If all participants are considered together, are there significant changes in participant behavior during the shadowing and post-test portions of the experiment? In other words, do the participants---as a group---change their behavior either while they are shadowing the recordings that I played for them or afterwards; and
        \item are there predictable relationships between speaker behavior in the baseline or post-test blocks and any of a set of linguistic, \hl{socio-psychological}, and demographic factors that have been measured as a part of this experiment? In short, can we explain how various factors contribute to the way that individuals participate in sound change, both within the context of this experiment and beyond it?
    \end{enumerate}
    
    I'm asking question (1)----whether speakers will first accommodate to a shadowing model and then carry that pattern of accommodation into a post-test---because if I can show empirically that speakers do both of these things in this order for a single variable, then that provides controlled, experimental evidence that the pattern of first adopting a feature from a donor dialect within the context of a single interaction and then carrying that feature over and continuing to use it in situations that do not involve the donor dialect, as described in Trudgill's \citeyearpar{trudgill1986dialects} theory of contact-induced dialect change, is correct. If, however, the theory is incorrect, then by attempting to reproduce the theoretical dynamic in the lab, we may gain some insight in to what exactly the \emph{true} dynamic is. This is the goal of \hl{this chapter}.
    
    As noted above, the theory is not borne out by the results described here. Question (2), then---whether and which of a number of factors investigated in this dissertation contribute to participation in real-world sound change---becomes important. It is of both theoretical and practical interest to be able to conduct laboratory studies of phenomena such as contact-induced dialect change and to explain what factors contribute to the way that people behave... at least, it's of interest if we're planning on being able to predict how sound change progresses through a society from person to person as Trudgill's theory suggests we should be able to do.
    
    Together, these chapters serve as a test of one aspect of \cbat{}, specifically that speakers must first engage in \sta{} and then transition into \lta{}. This chapter argues that \sta{} is not, in fact, a prerequisite for \lta{}, but rather that \sla{} are related in less obviously straighforward ways. Chapter \ref{microcosmchapter}, likewise, argues for the expansion of \cbat{}'s model of contact-induced dialect change from one whose sole explanatory variable is ``sufficient'' participation in interpersonal accommodation in interaction to one that includes a specific, broader set of explanatory variables. \hl{That's not his theory's sole varb. It also has at least a geographic and a demographic varb}
    
    %The quick and dirty answers to both of questions are ``yes'' and ``yes,''though it's not as simple as a two word answer might lead you to believe. Details below.