\chapter{Introduction}
\label{introchap}

\section{Outline}
\begin{enumerate}
    \item \ND{} is leveling towards Mandarin
    \item \emph{Why} is nanjing dialect leveling towards Mandarin? (one theory is that\ldots{})
    \item Can we replicate that theory's dynamic in the lab?
    \item Since it's more complicated than the theory would suggest, what might be contributing to participation in \lta{}?
\end{enumerate}

\section{\ND{} is leveling towards Mandarin}
    Nanjing, having had an estimated metro area population of roughly 11.7 million in 2010 \citep{oecd2015china}, is a mid-sized Chinese city. It is China's ``southern capital,'' and in addition to intermittently literally being the capital, has been an administrative and cultural center for many hundreds of years. Like many cities in China, Nanjing's distinct local dialect---\ND{}, a non-standard Mandarin dialect from the Jianghuai Mandarin dialect group \citep[p.179]{xu2006nanjing}---is undergoing marked contact-induced language change. Modern Standard Mandarin or $p\check{u}t\bar{o}nghu\grave{a}$, which I'll be calling simply ``Mandarin'' for short from here on out, has been exerting assimilatory pressure on \ND{} since at least the late 1950s \citep{bao1980sixty}. The dialects are partially mutually intelligible, though anecdotally, as a second language speaker of both dialects, at the point that I had been studying Mandarin long enough to understand 95\% of what was being said to me in that dialect, I could only understand maybe 50\% or 60\% of any Nanjing Dialect spoken to me. The dialects largely share syntactic structure and there is significant overlap in the lexicon \hl{(citations)}, and so subjectively the dialect differ mainly in their phonology. There are, however, a number of regular historical sound changes that have affected the dialects different\hl{(ial?)}ly, and so it is not \emph{too} terribly difficult to hear spoken Mandarin and reconstruct it into \ND{} or vice versa.
    
    \ND{}'s speakers are generally tellingly aware of a shift in speech norms away from those of the older generations, and specifically in the direction of Mandarin, and often relate that shift to age cohorts, the city's geography, and increased frequency of use of Mandarin at the expense \ND{}. For Example, during an exit interview at the end of the dissertation experiment, I told participants that I had heard that many Chinese dialects, \ND{} included, were gradually ``drawing close to'' ($k\grave{a}ol\check{o}ng$) Mandarin, and asked for their opinions on such changes. One participant replied that he and his friends ``don't use Nanjing dialect as much as we used to,'' and that they ``wouldn't use \ND{} as our first choice to speak to each other,'' unless it was to ``lighten the mood.'' Another replied that ``I've not only heard of this, I actually think this is a really salient phenomenon [\ldots{}] that is to say, Nanjing dialect is gradually deteriorating.'' A third participant rejected the premise of my question, asserting that regional pronunciations, such as those of ``California'' or ``New York,'' would never be ``changed away'' ($g\check{a}idi\grave{a}o$), and that ``Mandarin is only [...] perhaps influencing the pronunciation of some younger people,'' but that ``real \ND{} will never completely change into Mandarin. This is impossible.''
    
    \ND{}'s leveling towards Mandarin is, then, at once seen as both ``impossible'' and ``salient,'' and all the while the dialect is gradually losing ground as speakers' ``first choice'' for communication. Anecdotal evidence also suggests that some believe that younger Nanjinger's \ND{} is not ``authentic'' ($d\grave{\imath}dao$), or that you have to find older residents, particularly from the southern part of the city, if you want to hear ``genuine'' ($zh\grave{e}ngz\bar{o}ng$) \ND{}.
    
    I am not the first to study \ND{}, nor the effects of contact with Mandarin upon it. Empirical evidence for this process of ``$p\check{u}hu\grave{a}$'', or ``Mandarinization,'' also exists. Such evidence has been collected as shows Mandarinization in specific linguistic variables on the scale of decades by Bao \citeyearpar[orig. 1980]{bao1980sixty}, \cite{chen2011differences}, and myself \citeyearpar{coles2017mandarinization}, among others. Immediately below and in Chapter \ref{baselinechapter} I will describe some of the results of these studies with the goal of lending historical, linguistic, and broader social context to my synchronic, experimental, and perhaps less-than-entirely social results, to suggest that there are real parallels between what I find in this dissertation and how \cidc{} works \IRL{}.
    
    In an apparent-time study conducted in 1980 and republished in 2010, \citeauthor{bao1980sixty} looked at the progression of nine variables characteristic of Nanjing dialectal pronunciation and found that eight of those had shifted towards Mandarin norms at least to some degree since 1957, when both a ``dialect census'' and an earlier article of his own were published. Of those eight, four showed a \emph{complete} shift to Mandarin norms at least amongst the young. Similarly, in 2011 \citeauthor{chen2011differences} found that \hl{tones behaved way X, and speculated that certain PTH sandhi patterns were the cause}. Also, \cite{coles2017mandarinization} conducted an analysis of the \ND{} vowel inventory based on data from \cite{coles2014investigation} to show that younger Nanjingers had developed a distinction between the retroflex and apical \hl{fricated vowels} of Mandarin that their older Nanjinger compatriots lacked.
    
    The variables from \cite{chen2011differences} and \cite{coles2014investigation,coles2017mandarinization} are not investigated further here, but four of Bao's variables were taken up again in this study, and Chapter \ref{baselinechapter} will describe these and two other speech variables' relationships to both Bao's findings and my assertion that the \ND{} as a whole is Mandarinizing. \hl{set up what the actual varbs are so you can talk about relevance of other theories in next section.}
    
    I've long been personally concerned that such dialect leveling may lead to the eventual death-by-subsumption of language varieties in situations like \ND{}'s, resulting in a reduction of linguistic diversity locally and globally, and so this dissertation is in part meant to be a public service, to document 1) the current state of Nanjing dialect and the state of the national standard toward which it is leveling (Chapter \ref{baselinechapter}), 2) the process by which dialect leveling may take place (Chapter \ref{internalchapter}), and 3) the socio-psychological and linguistic factors that contribute to such leveling (Chapter \ref{microcosmchapter}). This project---which is a laboratory-based experimental study of the root causes of Mandarinization, specifically, but potentially reflective of dialect leveling anywhere---offers me the opportunity to comment on a widely used assumption in the dialect contact literature that such leveling proceeds necessarily first from interpersonal accommodation in interaction and then to long-term diffusion of a speech feature between individuals outside of contexts that might elicit continued accommodation. 
    
    It is true that the occurrence of dialect leveling does not necessarily imply that one dialect will eventually completely merge with another, and that it is entirely possible, or maybe even likely, that a new variety---an ``interdialect,'' to borrow a term from \citet{trudgill1986dialects}---will form as a result of contact between two mutually intelligible varieties.     It is also true to this day that ``Nanjing is the Jianghuai area's political, economic, and cultural center'' \citep[p.15]{bao1980sixty}, giving \ND{} its own cultural capital and cachet, but I do not believe that the state of the art has quite provided us with the means to reliably predict which of these paths \ND{} will eventually travel. However, given that China's central and local governments have long promoted Mandarin as the language of administration, business, and education (see, for example, Fig. \ref{fig:speakpth!}), and given that rates of active competence in Mandarin among city residents have been reported anywhere from roughly 30\% to more than 70\%, with passive competence as high as 97.2\% \citep{xu2006nanjing}, I wish to acknowledge that there is at least the possibility that \ND{} as a whole will lose its characteristic features as it levels towards the national standard.
        \begin{figure}
            \centering
            \includegraphics[scale=0.5]{figs/speakpth.jpeg}
            \caption{A poster celebrating the 40th anniversary of China's ``reform and opening,'' and exhorting viewers to ``Speak Mandarin well, enter a new era.'' Since 1988, China has conducted an annual ``Mandarin Promotion Week'' during the third week of September. Source: \cite{speakPTH!}.}
            \label{fig:speakpth!}
        \end{figure}
    It is in this context that I undertook the study described in this dissertation.
    
    While there are likely changes taking place on many levels---for example lexical, morphological, and syntactic---this dissertation is concerned only with sound changes currently under way in \ND{}, and specifically with those that are the result of contact with Mandarin. In general, when such changes are resulting in the reduction of differences between the two dialects on a large scale and over time and space, and so they may be referred to as ``dialect leveling.'' When such changes take place, however, within the context of a single, cross-dialectal, dyadic interaction, they are usually referred to as ``convergence'' \citep{giles1973accent} \hl{(OTHERS?)}, ``accommodation'' \citep{trudgill1986dialects} \hl{(OTHERS?)}, ``imitation'' \hl{(CITATION?)}, or more broadly ``interactional synchrony'' \hl{(CITATION? Look up citations halfway down page 2 of \textit{Dialects in Contact})}. I will generally use ``accommodation'' to refer to the interactional behavior and ``convergence'' to refer to the structural changes in a speaker's pronunciation that result from such behavior.
     
    I have noted that I am specifically interested in questions of contact-induced sound change, and so before going any further, I wish to clarify a point. In his review of sound change for \textit{The Routledge Handbook of Historical Linguitics}, Andrew Garrett discusses phonologization as the archetypal sort of sound change, but notes that there is some leeway in the concept such that scholars ``different researchers may find it useful to take a narrower or broader approach to the subject matter'' \citeyearpar[p.232]{garret2015sound}. \hl{Say more about how what is happening here is sound change, but also worth looking at for other-than-historical reasons.}

\section{\emph{Why} is \ND{} leveling towards Mandarin?}
    Theories of why sound change occurs abound, and range from the listener-focused to the interactional. For example, there's the idea that listeners innocently misapprehend a speaker's intention during listening and then faithfully try to reproduce what they \emph{believe} they've heard from their interlocutor \citep{ohala1981listener,ohala1989sound}, or the speaker-driven idea that change occurs due to frequency effects in phonological reduction (\hl{Bybee citations}). This dissertation's primary focus will be on an interactional theory, which posits that---in the context of dialect contact situations---change occurs as a result of an automatic tendency towards behavioral convergence between interlocutors \citep{trudgill1986dialects,auer2005role}, but the brief discussion that follows is of the relevance of other theories of sound change.
    
    In the 1980s, Ohala introduced a theory of sound change as misapprehension, wherein sound change occurs when listeners either over- or under-apply rules that would, if they functioned properly, factor out coarticulatory and other predictable ``noise'' that was not intended by the speaker as part of the ``signal.'' Ohala implies that it is not an unreasonable assumption that there are only ``a small number of accepted pronunciations'' for a given word, but states that even if this is true, there are nevertheless a ``seemingly unlimited number of measurably different phonetic variants of each word in actual speech'' \citeyearpar[p.179]{ohala1981sound}. In his view, for a listener to retrieve an invariant (or nearly invariant) phonological signal from this phonetic noise, they must have some experience-based rules for undoing the noisiness and retrieving the speaker's intent. When a rule is misapplied and a misapprehension results, the listener updates the phonological representation of the word in question to reflect the newly perceived pronunciation, and as the listener becomes the speaker it is (at least potentially) this new representation of the word that is selected as her own intended pronunciation.
    
    The question remains...\hl{why update? why select the updated pronunciation? What happens in cases of homonymic clash? I think Ohala must be implying that at a misapprehension, the new mistaken form becomes one of those ``small number of accepted pronunciations,'' but I haven't actually found a quote to that effect yet.}
    
    Beddor's study of the phonologization of vowel nasalization via covariation in temporal duration of the coarticulatory source (a nasal consonant) and its effect (nasalization on the preceding vowel) can be read to imply that sound change occurs when ``a gesture associated with a source segment comes to be reinterpreted as distinctively, rather than coarticulatorily, associated with a nearby [segment]'' \citeyearpar[p.785]{beddor2009coarticulatory}, i.e. effect is reinterpreted as source.
    
    \hl{Relevance of these theories to Nanjing Dialect's }\mdzn{}\hl{ }\IRL{}. These theories all have things to say with respect to common, phonetically motivated sound changes, but are not relevant for the changes that Nanjing dialect is undergoing, because these changes are 1) motivated by contact, and 2) do not involve the phonologization of phonetic effects. As an example, let's consider the changes to \ND{}'s nasal rhymes. As mentioned above, \ND{} has largely lost the nasal codas from its \textipa{/an/} and \textipa{/aN/} rhymes and replaced them with \textipa{[\~{\ae}]} and \textipa{[\~{A}]}, respectively, which Ohala would likely describe as a misattribution (or Beddor as a reattribution) of coarticulatory consonantal nasality and backness to the vowel itself. However, the interesting change from the perspective of this dissertation is not the nasalization of the vowel, but the development of any sort of contrast between the two rhymes. \cite[orig. 1980]{bao1980sixty} reports that in his 1957 study, only one out of 38 participants had any sort of distinction between these rhymes. In 1980, roughly half of 41 did. In my data, the contrast is pervasive throughout \ND{} and nasal codas following the \textipa{/\;A/} vowel have reappeared a small percentage of the time after being completely absent in Bao's two studies.

    The unmerging of two vowels and partial reintroduction of codas in exactly the same phonological context, affecting exactly the lexical sets of each rhyme as are affected in Mandarin, cannot easily be explained as a the reassignment of coarticulatory effects away from their sources within \ND{}, nor by phonological reduction due to frequency of use. The same logic applies to the introduction of more diphthongal realizations of \textipa{/iE/}, the shift from a preference for [l] over [n] to the opposite in situations where Mandarin has only [n], and the re-rounding of the /y/ vowel. The best explanation that we have for all of these phenomena, then, is that such changes are the result of contact with a dialect that has those features, namely Mandarin. Given that the changes underway in Nanjing are the product of contact with a dialect, and therefore not easily accounted for within the previously discussed theoretical frameworks, the theory to turn to is \cbat{}.
    
    In \textit{Dialects in Contact}, Peter Trudgill set out to explain how it could be that when mutually intelligible dialects come into contact with one another ``whole new language varieties, many of them eventually spoken by millions of people, grow and develop out of small-scale contacts between individual human beings'' \citep[p.161]{trudgill1986dialects}. Central to his argument was the idea that the mechanism for \cidc{} was the exchange of dialect features via interpersonal accommodation in cross-dialectal interaction at the level of individual speakers and listeners. \hl{Ref to synthesis?} Starting from \citeauthor{giles1973accent}' observation that individuals tend to ``reduce pronunciation dissimilarities'' \hl{(page?)} in interaction---provided they don't find their interlocutor offensive---Trudgill asserts that ``linguistic accommodation to salient linguistic features in face-to-face interaction is crucial in the geographical diffusion of linguistic innovations'' (pp. 82) because ``it is only during face-to-face interaction that accommodation occurs''(pp.40) and ``[i]f a speaker accommodates frequently enough to a particular accent or dialect, I would go on to argue, then the accommodation may in time become permanent'' (pp. 39). The implication is that frequent enough interaction with interlocutors using a dialect different from one's own will cause permanent change within a speaker's original dialect, which they in turn carry back to their home community and---under favorable enough conditions---spread to other speakers of their dialect. The book's primary concern, however, was describing the sorts of social, demographic, and phonological conditions that favor or inhibit either the geographical diffusion of dialectal features or the formation of interdialectal forms and varieties; Trudgill was less concerned with the specific process of the accommodative mechanism, and so the idea that accommodation between individuals must precede dialect leveling was taken as an assumption of the model rather than rigorously investigated. \hl{This paragraph includes spot ``D''}
    
    Several years later, Auer and Hinskens \citeyearpar{auer1996convergence,auer2005role} formalized the process described by Trudgill in their ``change-by-accommodation model’’ of \cidc{}, which they described in \citeyear{auer2005role} as progressing first from ``\sta{}'' in face-to-face interaction which ``may consist of either the adoption of [a] new feature and/or the abandonment of [an] older [one];’’ then to ``\lta{} as soon as it permanently affects the accommodating speakers;’’ and finally to the spread of the feature ``into the community at large‘’  \cite[p.335-6]{auer2005role}, additionally citing their \citeyear{auer1996convergence} paper as the source of their integrated, hierarchical model. Their \citeyear{auer2005role} chapter set out to bolster the theory with empirical evidence from a number of studies, but even here there was less support than you might imagine for the specific idea that ``\hl{[quote that says step 1 has to precede step 2]}’’ \hl{(citation)}. After considering evidence regarding correlations between the geographic spread of dialect features and the degree to which those features were subject to accommodation \citep{hinskens1992dialect}, as well as evidence from \cite{gilles1999dialektausgleich} that , Auer and Hinskens conclude that ``there is sufficient support for the hypothesis that levelling is foreshadowed in accommodation.''
    
    This conclusion is not, however, a straightforward endorsement of the position that accommodation \emph{must} precede leveling. \hl{Issues with Auer and Hinskens' approach.}
    
    %\hl{What do Auer and Hinskens mean ``foreshadowed?''} Hinskens says ``in our conception dialect levelling is foreshadowed in interactionally motivated variation, in accommodation towards the dialect background of the audience. This accommodation serves to reduce linguistic dissonance; we therefore expect gradually differing extents of accommodation in dialect use depending on the dialect variety spoken by audience members - other things being equal'' and that by his hypothesis he means ``the interactional, hence short-term phenomenon of accommodation in speech towards people with a different dialect background may lead to the structural focusing process of dialect levelling'' \citeyearpar[p.29]{hinskens1992dialect}. \hl{I *think* the problem is that they do the same thing that trudgill does and are like ``same constraints apply, must be the same process'' but don't test it.}

    Trudgill, in the process of building his argument,  defines geographic diffusion as having ``presumably'' taken place ``on the first occasion when a speaker employs a new feature in the \emph{absence} of speakers of the variety originally containing this feature'' \citeyearpar[emphasis in original]{trudgill1986dialects}, and Auer and Hinskens continue along these lines, but they abstract away somewhat from Trudgill's formulation in that they reduce the focus on geographical diffusion and apply their model more broadly to the \emph{interpersonal} spread of features. Rather than use the terms ``accommodation'' and ``diffusion,'' they use ``\sta{}'' and ``\lta{}'' to mean that accommodation that takes place in the presence of an interlocutor with a given speech feature and that accommodation that takes place after the interlocutor is no longer present. I will follow Auer and Hinskens in their use of these terms. \hl{I need to go read Auer and Hinskens 1992 to make sure that’s actually where the model shows up.}
    
    There are multiple causalities responsible for interactional convergence among interlocutors, including both cognitive and social underpinnings that push people either to use or to avoid using convergent behavior \citep{coles2017perspectives}. These underpinnings include the ``echoes'' of similar exemplars in memory activated by stimuli perceived, activated, and subsequently produced within the context of a single interaction \citep{goldinger1998echoes}, as well as decreased cognitive load due to the alignment of a listener's predictive conversation model with her interlocutor's actual behavior \citep{pickering2004toward,pickering2013integrated}. There are also biases towards macrosocial categories \citep{babel2010dialect} and even interactionally determined things like conversational role \citep{pardo2006phonetic} that can facilitate imitative behavior. Beyond all that, there are also factors that inhibit imitation, including \hl{factors + citations}. In this dissertation, such inhibiting and facilitating factors will be explored in Chapter \ref{microcosmchapter}, but earlier chapters (particularly Chapter \ref{internalchapter}) will be more concerned with the specifics of the progression from accommodation in interaction to the long-term adoption speech features in the absence of interlocutors who also have those features, as theories of \cidc{} have long proposed but rarely tested the idea that accommodation is an essential prerequisite for this type of change.
    
    \hl{The section above feels incomplete.}

    \hl{Fit these Ps in this section somewhere.}   \paragraph{Bidialectalism:}
    \cite{trudgill1986dialects} points out that in studies of language contact---where individuals must learn a new and mutually unintelligible variety in order to communicate---influence between varieties can often be attributed to imperfect learning or interference in the second language from the learner's first. He argues that dialect contact represents a special case, in that when the varieties in contact are already mutually intelligible, there is no need for individuals to learn each others' varieties, and indeed no strictly communicative impetus for them to do so. As a result, he uses
    
    \cite{xu2006nanjing} argues that Nanjing as a speech community maintains both Mandarin and \ND{} as locally acceptable codes for use in public interaction. He points to the fact that while only 9.8\% of speakers observed in his study used Mandarin with a known interlocutor uninvolved with the experiment, fully 97.2\% responded to Mandarin requests for directions, and 30.2\% actively used Mandarin to reply to such a request. His reported Mandarin response rate is low, compared to his \ND{} response rate (49.3\%), but he argues that given the fact that all requests for directions were in Mandarin the low Mandarin and high \ND{} response rates are an indication that Mandarin is seen as a local variety, and that a switch of code out of \ND{} is not necessary when you receive a request in Mandarin.
    
    this indicates a particularly high level of bidialectalism, as
    
    Trudgill's idea that ``new language varieties [\ldots{}] grow and develop out of small-scale contacts between individual human beings'' \citeyearpar[p.161]{trudgill1986dialects} is almost certainly right at a certain level\ldots{} Like, we all accept that languages can influence each other, and it's hard to imagine another way that they could do so, given that we're pretty sure listening to the TV doesn't change your accent\ldots{} but the logical leap that you \emph{must produce} altered versions of your own speech via accommodation \emph{while} you have an interlocutor present in order to change in the long-term is not well-founded.
    
    Given that sound change is a complex phenomenon, and that the sound changes going on in Nanjing are still more so, it is beneficial to bring as much of the situation as possible into the lab, where controlled experimental methodology may shed light upon the mechanisms involved. 

\section{Can we replicate that theory's dynamic in the laboratory?}
    Sound change of varying types has previously been reproduced to varying degrees in the laboratory. For example, as mentioned above, Ohala's theory of listener-driven sound change suggests that as soon as a listener has misapprehended a speaker's intent and internalized noise as if it were signal, sound change has occurred, and so laboratory studies that cause listeners to confuse two perceptually similar sounds are in and of themselves laboratory reproductions of sound change \citep{ohala1989sound}.\hl{this is weird, cuz if you're confusing p for b but it changes what word you think you're hearing, then how is that sound change? That's just a different word.} \hl{Relevance of Wright 1980, Ohala \& Amador 1981 (both cited in Ohala 1981)}
    
    \hl{copied from CH 4.} The observation that that short-term accommodation is prevalent in dyadic interaction \IRL{} was the foundation of ``Communication Accommodation Theory'' \citep{giles1973accent, giles1980accommodation, giles1987speech, giles1991contexts}, but it has since been discovered that such accommodative behavior is also fairly robust in the lab. Many scholars (e.g., \cite{pardo2006phonetic}, \cite{delvaux2007influence} \cite{nielsen2008word}, and \cite{zellou2016phonetic}) have shown accommodative convergence effects in experiments, including in less-social contexts like the shadowing paradigm used in this experiment. Long-term accommodation has also been convincingly observed both in the wild \hl{(CITATIONS, people adopting aspects of local accents in new countries that they move to:)}\citep{chambers1992dialect, kerswill2000creating} and in the laboratory \hl{(CITATIONS?)}. There's also plenty of evidence that not only will people change their speech significantly in all sorts of interactions, but that they'll accommodate across dialects \citep{delvaux2007influence,babel2010dialect,clopper2014sound}. And so, the assumption that \sla{} must be linked is reasonable, as is the hypothesis that it is a mechanism for contact-induced dialect change, but the questions of whether short-term accommodation \emph{leads} to long-term accommodation, and does so \emph{mechanistically} deserve to be carefully examined. 
    
    In these accounts, 
    
    Each of these scholar's work is quite different from the others, but all deal with issues of what causes sound change to occur, and do so using laboratory methods. Some of the potential benefits of studying sound change in the lab include:
    \begin{itemize}
        \item it introduces specific variables to which speakers can accommodate
        \item it yields a set time frame for measuring whether change has happened
        \item it takes less time and resources
        \item participants are less likely to drop out
        \item something\ldots{}
    \end{itemize}
    
    \cite{ohala1981listener} has some relevant stuff to say cuz he's all up in sound change. He says that the listener is not a passive participant in sound change, which I agree with. But he states that, at least in terms of phonetically motivated and ``universal'' sound changes, the listener participates in one of 3 ways:
    \begin{enumerate}
        \item by \emph{preventing} sound change by accurately filtering out coarticulatory noise;
        \item by underapplying rules to to filter out coarticulatory noise, thereby misinterpreting \emph{noise as signal}; or
        \item by overapplying rules to filter out coarticulatory noise, thereby misinterpreting \emph{signal as noise}
    \end{enumerate}
    \hl{stuff before this next sentence.} He also lists ``Wright 1980'' and ``Ohala and Amador 1981'' as examples of lab phon studies that have reproduced sound change in the lab.
    
    More recently, \cite{delvaux2007influence} is a good study to look at cuz it's basically a shadowing study where for a list of words (in carrier sentences?) they get baseline pronunciations; then play a regionally specific recording of each word twice over speakers and ask the participant to repeat the words; then they get post-test pronunciations. The important thing here is that they find convergence during the ambient speech portion, and that in the post-test ``speakers came back towards their own pronunciation, but usually not completely'' \hl{(p.25)}. This is \emph{almost} what I found, except that I found post-test effects \emph{without} finding shadowing effects (which I'll tell you all about in Chapter \ref{internalchapter}). But both point to long-term persistence of mental representations affecting pronunciation outside the context of the initial interaction. It's also interesting that we \emph{do} show complete regression back to pre-shadowing norms in the variables that got shadowed.
    
    Studies like \citeauthor{delvaux2007influence}'s---which include not only an experimental manipulation of individuals' pronunciation of a particular phonetic variable, but also include post-test examinations of the persistence of such changes into situations beyond the one that caused the initial change---are laboratory mirrors of \lta{} \IRL{}. 
    
    \cite{delvaux2007influence,nielsen2008word,pinget2015actuation,}\hl{(Ohala? Bybee? Beddor 2009 (and possibly 2013, which is less sound change-y, but still maybe related and helpful cuz ``info is used as it appears''), check out andrew garret's 2015 overview in the Routledge handbook of phonology. Rebecca sent a link with refs for like 5 overview chapters.) Include shadowing stuff. Talk about limitations of apparent time studies and say why my experimental pseudo-long-term approach is better. Basically cuz there should be less opportunity for anything else to go on, and cuz it's w/in subject. Set up ch }\ref{internalchapter}.

\section{It's more complicated than the theory suggests, so what might be contributing to participation in \lta{}?}
    Set up ch \ref{microcosmchapter}, people's estimates of effects \cite[?]{pinget2015actuation,sonderegger2012phonetic}

    Lots of people are interested in describing the factors that encourage individuals to participate in sound change, myself included.
    
    For example, in her investigation of two phonological mergers in Dutch, \cite{pinget2015actuation} finds that the degree to which speakers' perception of the changes is categorical, their attitudes towards the changes, and their imitative fidelity in a forced imitation task all significantly predict how advanced an individual is in these changes. Counter-intuitively, however, the specific correlation between imitative fidelity and advancedness of sound change is negative, such that \emph{worse} imitators are more advanced in the sound change. Most immediately, my interpretation of \cbat{} would lead me to assume that better imitators would be more advanced in a change-in-progress, as those with a strong ability to consciously imitate sounds might also be the individuals who imitate the most when they encounter new variants in the wild. A finding such as Pinget's is illustrative of the need to test such assumptions empirically.
    
    logic of the argument may be flipped such that the correlation is seen as evidence that poorer imitators are more advanced in a sound change. This idea is counter-intuitive in a ``\cbat{}'' context, given that the idea is that change comes from imitation; you'd expect better imitators to change more, and that is the opposite of both Pinget's findings, and my own. But that's exactly why these things should be tested. \hl{This next bit is maybe best in the discussion.} It's possible that these results are an artefact of measuring the wrong thing\ldots{} Those who are more advanced in a sound change towards some particular target will have less phonetic space in which to imitate towards the target, and so you may misinterpret a correlation that means ``people imitate, but like never all the way, regardless of how much they've already participated in sound change'' as meaning ``people who imitate are going to be worse sound changers.''

\cite{sonderegger2012phonetic} says \hl{something like} \ldots{} He uses ``long-term shifts'' to mean something more traditional, more stereotypically long-term, than my ``\lta{}'' (I guess basically permanent?) and in fact even his ``medium-term'' is longer than my ``long-term,'' where he means changes over like days and weeks and my is within the space of like 60 mins from the start of the experiment to the finish. He suggests there are varying levels of and qualitative types of ``time dependence'' \hl{(p.xv)} among both speakers and variables, which means
\begin{enumerate}
    \item that different people show different rates of change,
    \item that different people show different \emph{patterns} of change, and
    \item that both of the above are not only true of different people, but also of different \emph{variables}.
\end{enumerate}
These facts, he says, may be due in part to style. He also says that there is only ``a significant minority showing stability'' (\citeyearpar[p.xv-xvi]{sonderegger2012phonetic})

I think the important bits of sonderegger for Ch 5 are exactly those ideas listed above, which I think can be reasonably interpreted as ``if you're gonna predict what makes people or variables converge the way that they do, you're gonna need random effects for speaker... or to know which type of time dependence they use.''

\section{Important bits of theory}
\label{sec:theory}
Here are the important bits of theory that I need to piece together to tell a cohesive story about my data.

\subsection{Niedzielski and Giles 1996}
This publication is cited as saying that ``accommodation theory should be one of the major frameworks to which researchers in language change should turn'' \citep[p.335]{auer2005role}


\subsection{Trudgill}
\hl{This is less relevant than I used to think it was, but still possibly useful. I've cut a bunch of stuff that used to be here and stuck it at the end of the chapter.}

Here's what I currently think is important or useful about Trudgill:

\begin{itemize}
    \item assertion: ``the quantitative linguistic analysis of the accommodation process is a useful research tool.'' (pp. 37) \hl{duh}
    \item definition:  ``Diffusion can be said to have taken place, presumably, on the first occasion when a speaker employs a new feature in the \emph{absence} of speakers of the variety originally containing this feature'' (pp. 40, emph in orig) \hl{this is my long-term accommodation} \hl{this item is spot ``C''}
    \item assertion: ``linguistic accommodation to salient linguistic features in face-to-face interaction is crucial in the geographical diffusion of linguistic innovations'' (pp. 82) because ``it is only during face-to-face interaction that accommodation occurs''(pp.40) and ``[i]f a speaker accommodates frequently enough to a particular accent or dialect, I would go on to argue, then the accommodation may in time become permanent'' (pp. 39). \hl{poorly supported assumptions}
    \item ``We [\ldots{}] have very strong evidence that [varieties of colonial English] are as they are because of the way in which people behave linguistically in face-to-face interaction. That is, whole new language varieties, many of them eventually spoken by millions of people, grow and develop out of small-scale contacts between individual human beings'' (pp. 161). 
\end{itemize}

Here's what I think about that.

\begin{itemize}    
    \item first, the book is actually mostly about diachronic processes, and only briefly attempts to describe how synchronic accommodation works (Ch. 1), and he never actually tests whether accommodation in interaction is where dialect change comes from\ldots{} he just assumes it is at the beginning and then uses the assumption to create reasonable sounding explanations for his data.
    \item He \emph{does} occasionally explicitly assert that there is a direct link from short- to long-term accommodation (e.g., p. 39-40, 82, 145), but does not back it up well.
    \item His idea that ``new language varieties [\ldots{}] grow and develop out of small-scale contacts between individual human beings'' is almost certainly right at a certain level\ldots{} Like, we all accept that languages can influence each other, and it's hard to imagine another way that they could do so, given that we're pretty sure listening to the TV doesn't change your accent\ldots{} but the logic
\end{itemize}

\subsection{Auer and Hinskens 2005}
\begin{itemize}
    \item This may be where I got the terms \sla{}.
    \item Auer and Hinskens formalize the hierarchical 3 step model.
    \begin{itemize}
        \item ``\textit{1st component}. In face-to-face communication between speakers with more traditional speech habits and those who use an innovative form, the former accommodates to the linguistic behaviour of the latter[\ldots{}]However,[\ldots{}]it does not always have a lasting effect on the accommodating speaker’s linguistic `habits''' (pp. 335).
        \item ``\textit{2nd component}. Short-term accommodation becomes long-term accommodation as soon as it \emph{permanently} [my emphasis] affects the accommodating speakers'' (pp. 335). This apparently ``presupposes frequent exposure'' (pp. 336).
        \item ``\textit{3rd component}. The linguistic innovation may spread into the community at large, eventually leading to language change'' (pp. 336).
        \item (they also cite themselves from 1996 as saying this is the way it works)
    \end{itemize}
    \item Cites Labov 1990 (pp. 207) as saying that ``[e]ach act of communication between speakers is accompanied by a transfer of linguistic influence that makes their speech patterns more alike.'' 
    \item They say that \cite{trudgill1986dialects} has a ``particularly sophisticated version of the model'' \citep[p.336]{auer2005role}, an assessment with which I'm pretty sure I wholeheartedly disagree.
    \item Section 3 is the important one, cuz it says ``[t]he crucial point for the convergence-by-accommodation scenario is that \emph{accommodation is the first step to linguistic changes on the community level}'' (pp. 343, emphasis in original), underscoring the hierarchical nature of the model. 
    \item they say ``Accommodation may consist of either the adoption of the new feature,'' which they later term positive accommodation, ``and/or the abandonment of the older one(s),'' which they call negative accommodation.
    \item ``there is sufficient support for the hypothesis that levelling (3rd component) is foreshadowed in accommodation (1st component of the model discussed in section 1)'' (pp. 347-8) \hl{WAHT?!?! WHERE?!?!}
    \item They say that Gilles (1999, which I have so far only found in German) finds no (significant?) evidence of interdialectal accommodation in interaction, despite the fact that the dialects are (known to be?) converging.
    \begin{itemize}
        \item ``of the three speakers of the northern and eastern dialects in the sample who in an intradialectal conversation (i.e. with co-participants from the same area) still used a more than negligible number of the older forms at all, two (O1, O2) increased the use of the older forms in the interdialect condition, while only one (N1) decreased it (figure 13.3)'' (pp. 348). \hl{I don't know if this is relevant to my stuff, but seems like bad methods, cuz like if you only have 3 speakers who use it at all from a group of n speakers\ldots{} like, maybe the change is basically done and these people are just old fogeys that cling to the older style as a marker of identity or something. Maybe this is useful cuz I can be like ``look, I did a similar thing with way more people, and this is the pattern I got. 3 is too few.''}
    \end{itemize}
    \item ``Although in both cases (that of the Limburg dialects of Dutch and that of the Luxembourg dialects), an ongoing language change can be observed at the community level, individual speakers do not, or only in a few cases, accommodate to others who do not use the divergent dialect features (Hinskens), or who in their speech already use the spreading feature much more frequently or even exclusively (Gilles)'' (pp. 351)
    \item ``It seems relatively uncontroversial that long-term accommodation foreshadows change on the community level[\ldots{}]'' 
    \item ``After having discussed the results of some studies relevant for testing the validity of this [change-by-accommodation] model, we certainly cannot exclude the possibility that participants in interaction accommodate to each other’s behaviour, nor can we exclude the possibility that the frequency of exposure to a new, spreading, feature through intensive network contacts with its users can lead to the adoption of this variable. \emph{It has been difficult, however, to find evidence for the co-occurrence of interpersonal accommodation and community-level change}. \hl{yup}
    \item Several findings suggest that the driving force behind change in the individual, and also in the community, is not imitation of the language of one’s interlocutor but, rather, an attempt to assimilate one’s language to the possibly stereotyped characteristics of a group one wants to be part of, or resemble. \hl{except that doesn't work super well in my case. There's very little reason to want to resemble Annie in the post-test, and IAT and explicit attitudes are both only midling predictors of change.}
\end{itemize}

Here's what I think is important about this article:

\begin{itemize}
    \item This article as a whole doesn't seem super helpful
    \item the conclusion that it's ``difficult'' to ``find evidence for the co-occurrence of interpersonal accommodation and community-level change'' is actually sort of a non-conclusion with regards to that second step of the model. The attempt to find support for this idea in the Luxembourg study (Gilles, 1999) seems flawed to me cuz there are very few people who still use the old version at all, and those who do do not tend to converge with their new version-using interlocutors, and in fact occasionally diverge from them, which suggests to me that 1) the change is likely already essentially complete (i.e. it's difficult to even find the old variant), and 2) most of those who continue to use it are prolly doing so cuz they \textit{want} to project a dialectal identity (hence, divergence from the new version). If you don't find \sta{}, how can you test whether it leads to \lta{}?
    \item 
\end{itemize}

\subsection{Hinskens 1992}
\begin{itemize}
    \item The dissertation's ``hypothesis III'' seems relevant; ``Dialect levelling is foreshadowed in acconmodation'' That's from \hl{(p.26)}, then he says ``In other words, the interactional, hence short-term phenomenon of accommodation in speech towards people with a different dialect background may lead to the structural focusing process of dialect levelling'' \hl{(p.29)}. He points out Bell's thing that says people accommodate less to more distant audience members and says ``we therefore expect gradually differing extents of accommodation in dialect use depending on the dialect variety spoken by audience members - other things being equal.'' \hl{(p.29)}.
\end{itemize}
    I believe I finally get this argument. I think the overall point is that you \emph{should} see more or less dialect leveling depending on how much accommodation you get between speakers. \cite{auer2005role} cites this dissertation when talking about how if you have a more definitively local feature, you're more likely to drop it in conversation, and I guess the point is that dialects follow the same patter. Again, this is a little bit different from what I'm looking at cuz it makes a jump from the micro to the macro, and so they have to stop at ``foreshadowing,'' but I'm staying at the micro level to figure out if that interpersonal accommodation really is jumping to the macro-level. Only sorta.
    
\subsection{Kerswill 2002}
\begin{itemize}
    \item \cite[p.337]{auer2005role} says that this publication has a ``critical evaluation'' of revisions to the three-step model on page 680-682.
    \item \cite[p.351]{auer2005role} also cites this publication when it says ``[i]t seems relatively uncontroversial that long-term accommodation foreshadows change on the community level''
\end{itemize}
\subsection{Sonderegger}
So, Sonderegger (2012) makes some important points about the fact that for there to be change over time, change in the moment must \textit{persist}. He talks about the ``persistence hypothesis.'' This hypothesis basically says ``that short-term shifts which an individual makes during interaction can and do accumulate over time into long-term change'' \citep[p.102]{sonderegger2012phonetic}. Sonderegger finds support for a weakish form of the persistence hypothesis.

\section{Background on the IRL dialect situation}
In this section, I plan to expand upon the comments I’ve already made with respect to the nature of face-to-face interaction in Nanjing, and particularly to describe the nature of the diglossic situation that exists there today, relying on my interview data and the self-reported behavior of my speakers. There is also at least one source that deals with the subject of code choice in Nanjing \citep{xu2006nanjing}.

\pagebreak

\section{To Do:}
\begin{itemize}
    \item make sure there are empty argument brackets after all macros you made so that \LaTeX{} knows how to deal with the spacing.
    \item replace all the ind varb, dep varb, ling varb (and relevant lemmas) stuff with better, less confusing terminology (noticed re: ch 5).
    \item come up with a better term than ``generalization'' since it's not actually dependent on what they did in the shadowing block.
    \item take care of all the highlighted bits.
    \item decide whether ``mandarinization'' (and its lemmas) should be capitalized or not, and make it consistent throughout the dissertation. Also, check that it's spelled properly everywhere, cuz it's not in overleaf's dictionary.
    \item decide what your dep varbs, ind varbs, and ling varbs should be called (including whether they should be capitalized) and make sure your terminology is consistent throughout. Right now I think the variables themselves should be capitalized; ``Age'' would be the thing that predicts generalization, but ``age'' is a factor that's commonly cited as having an effect on participation in sound change (or whatever).
    \item deal with the fact that flipping the signs means different things for different variables in Ch 5.
    \item come up with better terminology to describe the predictors you use (BLP, IAT, etc.). ``Socio-psychological'' feels particularly wishy-washy.
    \item check for and standardize hyphenation of \begin{enumerate}
        \item ``pre-'' and ``post-test''
        \item ``hyper-standard''
        \item ``intra-'' and ``inter-personal''
    \end{enumerate}
    \item fix flipping of sign between BLP and IAT results (cuz BLP is + = pro ND, IAT is + = pro PTH).
    \item decide if you want to replace all (or perhaps some specific) ``pre-test'' mentions with ``baseline.''
    \item put in the ``donor'' and ''recipient'' dialect terminology throughout.
    \item standardize terminology for the manipulations; I noticed like 3 different names for the AN words in Ch \ref{internalchapter}.
    \item replace ``contact induced'' with ``contact-induced.''
    \item decide on ``mixed-effects'' vs ``mixed effects'' and standardize throughout.
    \item Ch 4's hypothesis list is spaced funkily after hyps 5 and 8. There's too much white space after.
    \item ``change-by-accommodation model’’ or ``\cbat{}’’?
    \item capitalization of ``\ND{}'' (but careful about things like ``Nanjing dialectal'')
    \item make sure this is true: ``I will generally use ``accommodation'' to refer to the interactional behavior and ``convergence'' to refer to the structural changes in a speaker's pronunciation that result from such behavior.''
    \item make sure all ``Nanjing Dialect'' instances (case sensitive) are ``\\ND{}''
    \item fix style file for URLs in bibliography
    \item make sure phoneme slashes don't linebreak weird anywhere.
\end{itemize}

\pagebreak

\section{Notes}

    One important bit of theory is Trudgill's (1986) theory about how mutually-intelligible dialects change when they come into contact with one another. The core idea (for the purpose of this dissertation) is that when two groups of people who speak different dialects interact over time, they undergo the following process:
    \begin{enumerate}
        \item Short-term accommodation
        \item Long-term accommodation
        \item Dialect change
    \end{enumerate}
    I'm going to describe what these mean now!

    So what Trudgill means by short-term accommodation is what is often referred to sort of generally in the \hl{socio-psychological} literature as "interactional synchrony." It's the process by which interlocutors come to generally behave--and more specifically, speak--like each other. So, for example, if I were speaking with Donald Trump Sr., I might start using grandiose terminology, or short, repetitive sentences (FIND CITATION ABOUT HOW TRUMP SPEAKS), mirroring elements of his syntax, word choice, and phonology, in addition to mimicking elements of his body posture.

    Long-term accommodation is when an individual has been speaking with someone else for long enough and frequently enough that they hold on to the features of their interlocutor's speech that they've picked up through short-term accommodation. That is, even after I'm no longer speaking to Donald Trump, I continue to use that same grandiose terminology, phonology, or posture.

    Dialect change, then, is what happens once I've taken these features of Mr. Trump's behavior, and carried them back to my home speech community. If after returning to my home town I continue to speak the way Mr. Trump does, it is possible that features of my speech that I have adopted from my encounter with Mr. Trump will then spread to other members of my community as I interact with them and they engage in acts of short-term accommodation with me. (This isn't really the idea, I think, of Trudgill's third step. The third step is actually maybe more just like when enough people have completed the second step, you can be like ``Look! The dialect has changed!'' but I'm not so concerned with that part of the problem.)

    The interesting thing here is that, as observed by \citet{auer2005role} ``[i]t [is] difficult[\ldots{}]to find evidence for the co-occurrence of interpersonal accommodation and community-level change.'' The issue I'm trying to address with this dissertation is that there doesn't really seem to be any research that looks at whether individuals actually engage in long-term accommodation. So, I've tried to design an experiment that would function as empirical evidence that long-term accommodation is a thing. (I should add something about the persistence hypothesis. Look at Sonderegger 2012 again and pull out any citations he has that indicate that short-term imitation can persist into the future even days after an initial interaction. \cite{goldinger2000role}, for example, says that listeners perceive speakers as being more similar to a recording that the speaker listened to a week earlier (I think), so like the speaker listened, imitated, and then a week later was still imitate-y sounding.) 

    What was the other theoretical stuff I thought was interesting a few weeks ago? Shit, I should have like written it on a scrap of paper or something. Maybe it had something to do with the framing I came up with for my LSA abstract. I'm gonna look at that now real quick.

    Big thoughts for this chapter

    Re: trudgill, auer & hinskens, & Pinget, each of these offers some nice framing, but doesn't quite do what I want to do, and so this is an opportunity to extend upon what they've done. I'm nuancing short term accommodation with better phonetic and psychological grounding, and looking much more explicitly at the mechanism by which we move from step 1 to step 2 of the hierarchical model.

    \cite{trudgill1986dialects} offers this nice, elegant idea; \cite{auer2005role} formalizes the model nicely and looks briefly at what I care about, and Pinget has a nice correlational approach to this (although there might be some more than that, I still need to read through it).

    R: focus on the ways that prior work hints at the things you care about, rather than being like "hey these studies suck."
    
    My dissertation is kind of different from Trudgill's 86 book and the studies from Giles that he references on pg 2 and 3 of that book, cuz I'm looking at dialects that are not necessarily differentiated by either social class or geographic mobility, tho numbers of speakers may be an issue.
    
    
    In this dissertation I will answer a number of questions of theoretical significance to those who may or may not be concerned about Nanjing, specifically, but are concerned with the mechanism by which dialects are so subsumed. In other words, the experiment I've conducted in Nanjing provides a valuable case study for looking at how language contact may lead to sound change.

    
    It was a tenet of the neogrammarian position that you cannot directly observe sound change \hl{fact check}. BUT! That didn't stop linguists from trying. 
    
    There's quite a history of attempts to bring sound change into the laboratory, including by leading lights such as John J. Ohala, Pam Beddor, and \hl{Joan Bybee [R: Lab?]}.
    
    As an example of the first of those situations, he makes the case that the seemingly mind-bending change from Proto-Indo-European labio-velars *ekwos and *gwiwos to Classical Greek hippos and bios is actually quite natural from an acoustic-perceptual point of view. He points out that the acoustic effect of a velar stop with lip rounding on vowel formant transitions is very similar to the effect of a bilabial stop \citep[p.182]{ohala1989sound}. As examples of the latter two situations, he 
    
    Ohala's assumption that ``listeners do their best to imitate the pronunciations they hear (or think they hear) in others' speech and thus adhere to the pronunciation norm'' \citeyearpar[p.191]{ohala1989sound} is relevant at the surface.
    
    Ohala's \citeyearpar{ohala1981listener} suggestion that the listener is not merely a passive participant in sound change also finds special relevance outside the context in which he intended its use, as there are factors which allow probabilistic prediction of a listener's propensities first for short- and then for \lta{}.
